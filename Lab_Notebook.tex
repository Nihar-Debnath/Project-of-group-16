\documentclass[a4paper,12pt]{article}
\usepackage{hyperref}
\usepackage{graphicx}
\usepackage{amsmath}
\usepackage{geometry}
\usepackage{fancyhdr}
\usepackage{enumitem}
\usepackage{tocbibind} 
\usepackage{ulem}
\usepackage{placeins}
\geometry{margin=1in}
\pagestyle{fancy}
\fancyhead[R]{\thepage}
\fancyhf{}

\title{
    \vspace{-1in}\textbf{Lab Notebook} \\
    \vspace{0.2in}\textbf{Semester:} II \\
    \textbf{Subject:} Software Tools and Technology-I Lab \\
    \textbf{Subject Code:} SEBCA1191\\
    
    \vspace{0.5in}
    \includegraphics[width=0.5\textwidth]{logo.png}
    \vspace{1in}
}

\author{
    \textbf{Group Number: 16} \\
    \textbf{Leader Name: Nihar Debnath}\\
    \textbf{Roll Number: 30001223008} \\
    \textbf{Department: B.C.A} \\
    \vspace{0.2in}\textbf{GitHub Repo Link:}
    \href{https://github.com/Nihar-Debnath/Project-of-group-16}{Click me}\\
    \begin{tabular}{|c|c|c|}
        \hline
        \textbf{Collaborators' Names} & \textbf{Roll Numbers} & \textbf{Department} \\
        \hline
        Abir Mandal & 30001223066 & B.C.A \\
        Anuksha Sarkar & 30084323018 & BSc in Data Science \\
        Ishita Das & 30084323011 & BSc in Data Science \\
        Sananda Choudhury  & 30001223004 & B.C.A \\
        \hline
    \end{tabular}
}
\date{}

\begin{document}

\maketitle
\newpage

\section*{Assignment Details}
\begin{itemize}
    \item \textbf{Assignment:} Create a Git Repository Containing Lab Notebook in a LaTeX File
\end{itemize}

\tableofcontents
\newpage

\section*{Leader: Nihar Debnath} 
\addcontentsline{toc}{section}{Leader: Nihar Debnath}
% Lab notebook content for Leader goes here
\date{\today}
\newpage

\section*{Member 1: Ishita Das}
\addcontentsline{toc}{section}{Member 1: Ishita Das}

\date{\today}
\FloatBarrier 
\begin{center}
\section*{\uline{-:GitHub:-}}
\end{center}

\paragraph{}
GitHub is a powerful platform for version control and collaborative software development, built on Git, a distributed version control system. It provides a comprehensive environment for managing code repositories, allowing developers to track changes, collaborate on projects, and maintain different versions of their code. Through features like branches and pull requests, GitHub facilitates seamless teamwork by enabling parallel development and peer review of changes before they are integrated into the main project. Additionally, GitHub's issue tracking system helps manage and prioritize tasks, while GitHub Actions offers automation for continuous integration and delivery. With its user-friendly interface and robust toolset, GitHub has become an essential resource for developers, fostering efficient workflows and innovation in the software development community.


\begin{center}
\section*{\uline{:Installation:}}
\end{center}

\paragraph{}
Installing GitHub Desktop is a straightforward process that begins with downloading the application from the official GitHub Desktop website. For Windows users, after downloading the .exe installer, you simply run it and follow the on-screen instructions to complete the installation. The process includes selecting installation options and agreeing to the terms of use. For macOS users, you download the .dmg file, open it, and drag the GitHub Desktop icon into the Applications folder to install. Once installed, you can launch the application, where you'll be guided through initial setup steps such as signing in with your GitHub account and configuring basic Git settings. This setup enables you to efficiently manage your repositories and collaborate on projects using the intuitive graphical interface provided by GitHub Desktop.

\begin{figure}[h!]
    \centering
    \includegraphics[width=0.5\linewidth]{GitHub-Symbol.png}
    \caption{GITHUB}
    
\end{figure}


\newpage

\section*{Member 2: [Anuksha Sarkar]}
\addcontentsline{toc}{section}{Member 2: [Anuksha Sarkar]}
% Lab notebook content for Member 2 goes here
\date{\today}
\FloatBarrier 
\begin{center}
\section*{\uline{-:What is the purpose of the calculator program:-}}
\end{center}

\paragraph{}
The primary purpose of a calculator program is to perform mathematical calculations efficiently and accurately.** It provides a user-friendly interface for entering numbers and operators, and then quickly calculates and displays the result. 

Calculators can range from simple devices that handle basic arithmetic to more advanced ones capable of complex functions like trigonometry, statistics, and even programming. They are widely used in various fields, including education, science, engineering, and everyday life.



\begin{center}
\section*{\uline{:Explain the code properly:}}
\end{center}

\paragraph{}

Purpose:
This C program is designed to function as a simple calculator, allowing users to perform basic arithmetic operations (addition, subtraction, multiplication, and division).\\

Key Components and Their Functions:\\

1. Header Inclusion:
   - $#include <stdio.h>$: This includes the standard input/output library, which provides functions like printf (for printing output) and scanf (for reading input).\\

2. Main Function:
   - int main(): This is the entry point of the program. It's where the execution begins.\\

3. Variable Declarations:
   - char operator: Stores the operator entered by the user (+, -, *, /).
   - double num1, num2: Stores the two numbers entered by the user.
   - double result: Stores the calculated result.\\

4. User Input:
   - printf("Enter an operator (+, -, *, /): ");: Prompts the user to enter an operator.
   - scanf("%c", &operator);: Reads the character entered by the user and stores it in the operator variable.
   - Similar prompts and reads are used for the two numbers.\\

5. Operator Checking and Calculation:
   - switch (operator): This is a conditional statement that checks the value of the operator variable.
   - case '+':, case '-':, case '*':, case '/':: These cases handle different operators.
   - The respective arithmetic operations are performed based on the operator and the results are stored in the result variable.
   - A check for division by zero is included to prevent errors.\\

6. Output:
   - `printf("Result: ")\\

Flow of Execution:\\

1. The program starts execution from the main function.
2. The user is prompted to enter an operator.
3. The entered operator is stored.
4. The user is prompted to enter two numbers.
5. The entered numbers are stored.
6. The switch statement checks the operator and performs the corresponding calculation.
7. The result is printed.
8. The program ends.\\
Overall, this code provides a basic calculator functionality by taking user input for an operator and two numbers, performing the calculation based on the operator, and displaying the result.

\begin{center}
\section*{\uline{:How to use it, what will be the output after program compilation, set an example:}}
\end{center}

\paragraph{}
How to Use:\\

1. Compile the code:
   - Save the code in a .c file (e.g., calculator.c).
   - Open a terminal or command prompt.
   - Navigate to the directory where you saved the file.
   - Compile the code using a C compiler like GCC:
     gcc calculator.c -o calculator
     This will create an executable file named calculator.\\

2. Run the program:
   - Execute the executable file:
     ./calculator

Example:
Enter an operator (+, -, *, /): +
Enter two numbers: 5 3
Result: 8.00\\

In this example:
- The user enters + as the operator.
- The user enters 5 and 3 as the numbers.
- The program calculates 5 + 3 and prints the result 8.00.\\

Other examples:
- - as the operator: 5 - 3 will result in 2.00.
- * as the operator: 5 * 3 will result in 15.00.
- / as the operator: 5 / 3 will result in 1.67.

\newpage

\section*{Member 3: [Sananda Choudhury]}
\addcontentsline{toc}{section}{Member 3: [Sananda Choudhury]}
% Lab notebook content for Member 3 goes here
\date{\today}
\FloatBarrier 
\begin{center}
\section*{\uline{-:Converting a submit button to "chin tapak dum dum":-}}
\end{center}
\paragraph{}

#Steps to Fix the Button and Create a Pull Request\\

1. *Clone the Repository:*
   - You’ve already cloned the repository using GitHub Desktop, so you should have the project on your local machine.\\

2. *Open the Project:*
   - Open the project in your preferred IDE as per the instructions in the readme.md.\\

3. *Locate the Button Code:*
   - Search for the code where the submit button is defined. This is typically found in the frontend code of the application. Depending on the technology stack used (e.g., HTML/CSS, React, Angular, etc.), it could be in a file like index.html, App.js, ButtonComponent.js, or a similar file.\\

4. *Rename the Button:*
   - You renamed the button to "Chin Tapak Dum Dum". If the button’s size became disproportionate, it’s likely due to styling issues.\\

5. *Test the Application:*
   - Run the application again to ensure that the button appears correctly and that there are no additional issues.\\

6. *Commit Your Changes:*
   - Once the button looks good, commit your changes. Use descriptive commit messages, for example: Fixed button styling after renaming.

   bash
   git add .
   git commit -m "Fixed button styling after renaming to 'Chin Tapak Dum Dum'"
   \\
   
7. *Push Your Changes:*
   - Push your changes to your forked repository on GitHub.

   bash
   git push origin main
   

   (Replace main with the correct branch name if it's different.)\\
   
8. *Create a Pull Request:*
   - Go to the original repository on GitHub (the one you cloned from).
   - You should see an option to create a pull request from your forked repository. Follow the instructions to create a pull request with a title and description of what you have done.

   Make sure to mention in the pull request that you have fixed the button styling after renaming it.


\newpage

\section*{Member 4: [Member 4 Name]}
\addcontentsline{toc}{section}{Member 4: [Member 4 Name]}
% Lab notebook content for Member 4 goes here
\date{\today}

\end{document}