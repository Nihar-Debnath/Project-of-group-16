\documentclass[a4paper,12pt]{article}
\usepackage{hyperref}
\usepackage{graphicx}
\usepackage{amsmath}
\usepackage{geometry}
\usepackage{fancyhdr}
\usepackage{enumitem}
\usepackage{tocbibind} 
\usepackage{ulem}
\usepackage{placeins}
\geometry{margin=1in}
\pagestyle{fancy}
\fancyhead[R]{\thepage}
\fancyhf{}

\title{
    \vspace{-1in}\textbf{Lab Notebook} \\
    \vspace{0.2in}\textbf{Semester:} II \\
    \textbf{Subject:} Software Tools and Technology-I Lab \\
    \textbf{Subject Code:} SEBCA1191\\
    
    \vspace{0.5in}
    \includegraphics[width=0.5\textwidth]{logo.png}
    \vspace{1in}
}

\author{
    \textbf{Group Number: 16} \\
    \textbf{Leader Name: Nihar Debnath}\\
    \textbf{Roll Number: 30001223008} \\
    \textbf{Department: B.C.A} \\
    \vspace{0.2in}\textbf{GitHub Repo Link:}
    \href{https://github.com/Nihar-Debnath/Project-of-group-16}{Click me}\\
    \begin{tabular}{|c|c|c|}
        \hline
        \textbf{Collaborators' Names} & \textbf{Roll Numbers} & \textbf{Department} \\
        \hline
        Abir Mandal & 30001223066 & B.C.A \\
        Anuksha Sarkar & 30084323018 & BSc in Data Science \\
        Ishita Das & 30084323011 & BSc in Data Science \\
        Sananda Choudhury  & 30001223004 & B.C.A \\
        \hline
    \end{tabular}
}
\date{}

\begin{document}

\maketitle
\newpage

\section*{Assignment Details}
\begin{itemize}
    \item \textbf{Assignment:} Create a Git Repository Containing Lab Notebook in a LaTeX File
\end{itemize}

\tableofcontents
\newpage

\section*{Leader: Nihar Debnath} 
\addcontentsline{toc}{section}{Leader: Nihar Debnath}
% Lab notebook content for Leader goes here
\date{\today}
\newpage

\section*{Member 1: Ishita Das}
\addcontentsline{toc}{section}{Member 1: Ishita Das}

\date{\today}
\FloatBarrier 
\begin{center}
\section*{\uline{-:GitHub:-}}
\end{center}

\paragraph{}
GitHub is a powerful platform for version control and collaborative software development, built on Git, a distributed version control system. It provides a comprehensive environment for managing code repositories, allowing developers to track changes, collaborate on projects, and maintain different versions of their code. Through features like branches and pull requests, GitHub facilitates seamless teamwork by enabling parallel development and peer review of changes before they are integrated into the main project. Additionally, GitHub's issue tracking system helps manage and prioritize tasks, while GitHub Actions offers automation for continuous integration and delivery. With its user-friendly interface and robust toolset, GitHub has become an essential resource for developers, fostering efficient workflows and innovation in the software development community.


\begin{center}
\section*{\uline{:Installation:}}
\end{center}

\paragraph{}
Installing GitHub Desktop is a straightforward process that begins with downloading the application from the official GitHub Desktop website. For Windows users, after downloading the .exe installer, you simply run it and follow the on-screen instructions to complete the installation. The process includes selecting installation options and agreeing to the terms of use. For macOS users, you download the .dmg file, open it, and drag the GitHub Desktop icon into the Applications folder to install. Once installed, you can launch the application, where you'll be guided through initial setup steps such as signing in with your GitHub account and configuring basic Git settings. This setup enables you to efficiently manage your repositories and collaborate on projects using the intuitive graphical interface provided by GitHub Desktop.

\begin{figure}[h!]
    \centering
    \includegraphics[width=0.5\linewidth]{GitHub-Symbol.png}
    \caption{GITHUB}
    
\end{figure}


\newpage

\section*{Member 2: [Member 2 Name]}
\addcontentsline{toc}{section}{Member 2: [Member 2 Name]}
% Lab notebook content for Member 2 goes here
\date{\today}
\newpage

\section*{Member 3: [Member 3 Name]}
\addcontentsline{toc}{section}{Member 3: [Member 3 Name]}
% Lab notebook content for Member 3 goes here
\date{\today}
\newpage

\section*{Member 4: [Member 4 Name]}
\addcontentsline{toc}{section}{Member 4: [Member 4 Name]}
% Lab notebook content for Member 4 goes here
\date{\today}

\end{document}
